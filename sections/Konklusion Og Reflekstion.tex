\chapter{Konklusion Og Reflektion}\label{ch:KonklusionReflektion}

\section{Konklusion}
%Problemformulering
%Hvordan bygger vi bedst et system, som kan dokumentere ens rejseoplevelser på en overskuelig måde og sætter de minder man deler i bedre kontekst, samt hvordan håndterer vi bedst muligt at flere brugere kan redigere i samme tekst under hver sektion af landkortet?
Målet med projektet var at udvikle et system som kunne bruges til at dokumentere rejseoplevelser, og skulle håndtere samtidighed mellem brugere. 
%Agile metode vs UP
%XP
%Centrale artefakter - TDD, Pair programming, Small releases, Continuos integration, Refactoring
%Risici
%Arkitektur og kvalitetsikring

Sammenlignet med et tidligere project med UP og plan-driven udvilkning, kan gruppen konkludere, at den agile udviklingsmetode har været fordelagtig. Kombinationen af Scrum og XP har vist sig, at denne udviklings tilgang har været nyttig til at give et overblik over systemets udvikling fra start til slut. Daily Scrum og 2 ugers Sprints har hjulpet gruppen i en høj grad med ikke at komme ud af kurs, derved undgå risici. 

Elementerne fra XP skabte under Scrum, så som TDD, Pair Programming, et bedre fundament gennem udviklingen i hverdagen. 



\section{Reflektion}