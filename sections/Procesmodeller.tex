\chapter{Processmodeller}\label{ch:processmodeller}

\section{Den agile metode}
I slut 90’erne blev den agile metode udviklet, grundet for dette, var at der var stor utilfredshed med den daværende fremgangsmetode til udvikling af softwaresystemer. Før den agile metode blev indført, var den generelle overbevisning, at den bedste måde at opnå bedst mulig software på, var ved at planlægge det forsigtigt. Dette syn på udviklingsmetoden kom fra software ingeniør samfundet, som for det meste udviklede software i store teams der arbejdede for forskellige firmaer. Meget af dette software blev brugt til store systemer, så som flysystemer eller til systemer til staten. Derfor krævede processen meget detaljeret planlægning på forhånd, og mere tid blev sat af til planlægningen om hvordan systemet skulle udvikles end selve udviklingen eller test. 
Det så ud til, at denne udviklingsmetode virkede meget fint, når det galte at udvikle software for store firmaer, men hopper vi ned til de ikke så store firmaer eller små forretninger, vil den tunge planlægningsmetode ikke være den mest optimale udviklings metode. Med den nye metode, fokuserede man stedet på software, i stedet for selve designet og dokumentationen af systemet. Denne fremgangsmåde var nemmere, når det kom til mindre systemer, da  udviklingsmetoden og kravenede hurtigt ændre sig gennem processen. \cite{Sommerville}
Så, utilfredsheden med den tunge planlægningsmetode ledte til udviklingen af agile metoder, så som Extreme Programming og Scrum. 


\textbf{Hvad er den agile metode så?} 

Agile metoder er Rapid software udvikling, det er metoder der er designet til at producerer software der ikke bare er blevet produceret hurtigt, men nyttigt. \cite{Sommerville} Når det gælder agile metoder, er kunderne en del af udviklingsprocessen, dette kommer i form af de små increments der konstant bliver udviklet og frigjort til kunden. Dette skaber en hurtig feedback tilbage til teamet, så teamet ved hvad der skal forbedres, eller hvis der sker nye forandringer ift. krav. Dette betyder, at teamet gerne skal have feedback, hurtigst muligt, og derfor minimasere vi dokumentation, ( som også er en af målene for agile metoder ) og sætter i stedet fokusset på informativt kommunikation. Hvorimod ved plan-driven sker kommunikationen mellem stages og med formal dokumentation. De to udviklingsmetoder behøves ikke skilles fra hinanden fuldstændigt, agile metoder kan sagtens benyttes med plan-driven. \cite{Sommerville}
I den tiende version af Software Engineering, af Ian Sommerville, er det specificeret at agile metoder har været succesfulde for to udviklings strategier.

Den første fortæller, at agile har været succesfuld, når det gælder både små produkter til medium størelse af produkter, og at stort set alle software produkter og applikationer er udviklet ved brug af en agile tilgang. 

Den anden, fortæller os, at agile metoder specielt har været funktionelt, da kunden ofte er  en del af udviklingsprocessen. Samt, få eksterne stakeholders, der har en effekt på softwaren. \cite{Sommerville}
Givet disse situationer, har agile metoder fungeret perfect. Grundet dette, ligger hos at kommunikationen mellem stakeholders er konstant og uformel. Udover, er softwaret ikke påvirket af en masse tidligere stillet faste krav. 

Tidligere blev det nævnt, at agile metoder ikke altid har været mainstream-metoden for software udvikling. Et af de første udvilingsmetoder har været The Waterfall Method.

\section{The Waterfall model}
De første software udviklingsmodeller omhandlede en proces der var struktureret af et antal faser, der repræsenterede hvordan udviklingen af software skulle forekomme. \cite{Sommerville}



På Figur1 ses det, at prosessen har en vandfald lignende arkitektur, og herfra kommer navnet for The Waterfall Model. Princippet bag modellen er, at teamet planlægger fremadrettet for alle aktiviteterne, før udviklingen af softwaren begynder. Herunder, ses det at vandfaldsmodellens faser reflekter de forskellige aktiviteter i traditionel software udvikling. \cite{Sommerville}
\begin{itemize}
    \item Krav analyse og definition 
    \item System and software design
    \item Implantation og unit test
    \item Integration og system test 
    \item Operationer og vedligeholdelse 
\end{itemize}

Når det gælder vandfalsmodellen, så bør man ikke gå videre til næste fase, før den tidligere fase er færdiggjort og dokumenteret. Problemet ved at gå til software udvikling med denne model, er at gennem udviklingen af et helt projekt vil der altid være et tidligere problem at identificere. Hvad der menes med dette er, at hvis f.eks et team der er i gang med design, vil hurtigt identificere problemer med krav, da det ofte ikke stemmer overens. Til gengæld, vil denne ikke være dårlig for hardware udvikling. Så, nyt information vil altid komme frem, i hver aktivitet. Det betyder, at hvis noget viser sig at være for et kompleks at lave, uden feedback fra kunden, kan der være store forsinkelser i software designet. Dette kan lede til et dårligt design. \cite{Sommerville}

\section{Metode Valg}

Nu hvor 2 metoder kort er blevet beskrevet, kan der på gruppen vurderes, om der skal udvikles i vandfalds metoden eller en agil metode. Før dette, er det vigtigt at pointere, at gruppen har en begrænset tidsperiode på lidt over 2 måneder til at færdig gøre projektet, som omhandler en web applikation og en konsol applikation. 

Eftersom, at teamet kun er på 4 personer og dette er en relativ lille applikation, som skal udvikles, er det besluttet, at applikationen bliver udviklet med agile metoder. Dette er i sig selv ikke grund nok, for at vælge netop det agile over vandfalds metoden. Derfor, er det vigtigt at gå over begge metoder, dette vil hjælpe bedre med at forstå, hvorfor agile metoder skal vælges over vandfalds metoden, specielt gældende dette projekt. Et af de fundamentale grundlag for at vælge agile imodsætning til vandfalds, er fordi applikationen er et socialt medie, som operer på cloud-baseret-software eller bedre kendt som software as a service (saas). Saas er en metode, hvori software tillader dataens tilgængelighed til alle enheder med internet adgang og en web browser.\cite{SoftwareAdvice} Saas modeller vil aldrig virke med vandfalds baseret udvikling, da det aldrig kan blive færdiggjort, fordi det nettop kræver konstant vedligeholdelse og opdateringer. 

Som nævnt tidligere, og også beskrevet i Software Engineering af Ian Sommer, så benyttes agile metoder i stort set alt software udvikling nu om dage, samt er langt de fleste apps udviklet ved brug af agile tilgange. \cite{Sommerville} Kunden for dette projekt er ikke en ekstern kunde heller, men teamet selv, som også er involveret i udviklingen af softwaren og har derfor noget at sige. I denne situation, hvor kunden er selve teamet, er det muligt at have en konstant kommunikation mellem kunden, projekt lederen og udviklingsteamet. Siden kunden er teamet selv, vil der være sandsynlig mere feedback gennem processen. Teamet er et lille team, og med konstant feedback vil det være nemmere at indføre konstant ændringer i kravene og softwaren til systemet. Dette skulle gerne gøres, uden at gå tilbage og lave dokumenterende ændringer, før en ændring i softwaren kan indføres. Så, med den agile model kan teamet hurtigt justere til kundes behov.  



