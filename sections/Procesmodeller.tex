\chapter{Processmodeller}\label{ch:processmodeller}

\section{Den agile metode}
I slut 90’erne blev den agile metode udviklet, grundet for dette, var at der var stor utilfredshed med den daværende fremgangsmetode til udvikling af softwaresystemer. Før den agile metode blev indført, var den generelle overbevisning, at den bedste måde at opnå bedst mulig software på, var ved at planlægge det forsigtigt. Dette syn på udviklingsmetoden kom fra software ingeniør samfundet, som for det meste udviklede software i store teams der arbejdede for forskellige firmaer. Meget af dette software blev brugt til store systemer, så som flysystemer eller til systemer til staten. Derfor krævede processen meget detaljeret planlægning på forhånd, og mere tid blev sat af til planlægningen om hvordan systemet skulle udvikles end selve udviklingen eller test. 
Det så ud til, at denne udviklingsmetode virkede meget fint, når det galte at udvikle software for store firmaer, men hopper vi ned til de ikke så store firmaer eller små forretninger, vil den tunge planlægningsmetode ikke være den mest optimale udviklings metode. Med den nye metode, fokuserede man stedet på software, i stedet for selve designet og dokumentationen af systemet. Denne fremgangsmåde var nemmere, når det kom til mindre systemer, da  udviklingsmetoden og kravenede hurtigt ændre sig gennem processen. \cite{Sommerville}
Så, utilfredsheden med den tunge planlægningsmetode ledte til udviklingen af agile metoder, så som Extreme Programming og Scrum. 


\textbf{Hvad er den agile metode så?} 

Agile metoder er Rapid software udvikling, det er metoder der er designet til at producerer software der ikke bare er blevet produceret hurtigt, men nyttigt. \cite{Sommerville} Når det gælder agile metoder, er kunderne en del af udviklingsprocessen, dette kommer i form af de små increments der konstant bliver udviklet og frigjort til kunden. Dette skaber en hurtig feedback tilbage til teamet, så teamet ved hvad der skal forbedres, eller hvis der sker nye forandringer ift. krav. Dette betyder, at teamet gerne skal have feedback, hurtigst muligt, og derfor minimasere vi dokumentation, ( som også er en af målene for agile metoder ) og sætter i stedet fokusset på informativt kommunikation. Hvorimod ved plan-driven sker kommunikationen mellem stages og med formal dokumentation. De to udviklingsmetoder behøves ikke skilles fra hinanden fuldstændigt, agile metoder kan sagtens benyttes med plan-driven. \cite{Sommerville}
I den tiende version af Software Engineering, af Ian Sommerville, er det specificeret at agile metoder har været succesfulde for to udviklings strategier.

Den første fortæller, at agile har været succesfuld, når det gælder både små produkter til medium størelse af produkter, og at stort set alle software produkter og applikationer er udviklet ved brug af en agile tilgang. 

Den anden, fortæller os, at agile metoder specielt har været funktionelt, da kunden ofte er  en del af udviklingsprocessen. Samt, få eksterne stakeholders, der har en effekt på softwaren. \cite{Sommerville}
Givet disse situationer, har agile metoder fungeret perfect. Grundet dette, ligger hos at kommunikationen mellem stakeholders er konstant og uformel. Udover, er softwaret ikke påvirket af en masse tidligere stillet faste krav. 

\section{The Waterfall model}
De første software udviklingsmodeller omhandlede en proces der var struktureret af et antal faser, der repræsenterede hvordan udviklingen af software skulle forekomme. \cite{Sommerville}


På Figur1 ses det, at prosessen har en vandfald lignende arkitektur, og herfra kommer navnet for The Waterfall Model. Princippet bag modellen er, at teamet planlægger fremadrettet for alle aktiviteterne, før udviklingen af softwaren begynder. Herunder, ses det at vandfaldsmodellens faser reflekter de forskellige aktiviteter i traditionel software udvikling. \cite{Sommerville}
\begin{itemize}
    \item Krav analyse og definition 
    \item System and software design
    \item Implantation og unit test
    \item Integration og system test 
    \item Operationer og vedligeholdelse 
\end{itemize}

 