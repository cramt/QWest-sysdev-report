\chapter{Processmodeller}\label{ch:processmodeller}

\section{Den agile metode}
I slut 90’erne blev den agile metode udviklet\cite{Sommerville}. Den blev udviklet som modsvar på den utilfredshed der var med den daværende fremgangsmetode til udvikling af softwaresystemer. Før den agile metode blev indført, var den generelle overbevisning, at den bedste måde at opnå bedst mulig software på, var ved at planlægge det forsigtigt\cite{Sommerville}. Dette syn på udviklingsmetoden kom fra software ingeniør fællesskabet, som for det meste udviklede software i store teams der arbejdede for forskellige firmaer. Meget af dette software blev brugt til store systemer, såsom flysystemer eller til systemer til staten. Derfor krævede processen meget detaljeret planlægning på forhånd, og mere tid blev sat af til planlægning af hvordan systemet skulle være designet end selve udviklingsprocessen eller test\cite{Sommerville}.
Det så ud til, at denne udviklingsmetode virkede fint når det gjaldt udvikling af software til store firmaer, men angående de mindre firmaer eller små forretninger ville den tunge planlægningsmetode ikke altid være den optimale udviklingsmetode. Med den nye metode, fokuserede man på software og hvordan løsningen og designet kan ændre sig efterhånden som man bliver klogere på problemerne der skal løses, samt ny teknologi der konstant udkommer, i stedet for selve designet og dokumentationen af systemet. Denne fremgangsmåde var nemmere, når det kom til mindre systemer, da udviklingsmetoden og kravenede hurtigt ændrede sig gennem processen\cite{Sommerville}. 
Så, utilfredsheden med den tunge planlægningsmetode ledte til udviklingen af agile metoder, såsom eXtreme Programming og Scrum.


\textbf{Hvad er den agile metode så?}

\begin{figure}
    \includegraphics[width=\linewidth]{figures/AgileMetoder.png}
    \caption{Den Agile metode}
    \label{fig:Agil}
\end{figure}

Agile metoder er Rapid software udvikling. Det er metoder som er designet til at producere software, der ikke bare er blevet produceret hurtigt, men nyttigt\cite{Sommerville}. Når det gælder agile metoder, er kunderne en del af udviklingsprocessen. Dette kommer i form af de små increments der konstant bliver udviklet og frigjort til kunden. Dette skaber en hurtig feedback tilbage til teamet, så teamet ved hvad der skal forbedres, eller hvis der sker nye forandringer ift. krav. Dette betyder, at teamet gerne skal have feedback hurtigst muligt, og derfor minimerer vi dokumentation (som også er en af målene for agile metoder) og sætter i stedet fokus på informativ kommunikation. Ved plan-dreven udvikling sker kommunikationen mellem stages og med formel dokumentation. De to udviklingsmetoder behøves ikke skilles fra hinanden fuldstændigt, agile metoder kan sagtens benyttes med plan-driven. Agile metoder har været succesfulde for to udviklings strategier;
For det første har agile metoder været succesfulde, når det gælder både lille til medium produktstørrelse, og at stort set alle software produkter og applikationer er udviklet ved brug af en agile tilgang.
For det andet har agile metoder været specielt fordelagtige, eftersom kunden også nu har en vigtig rolle, og deltager i udviklingsprocessen. Derudover er der også få eksterne interessenter, der har indflydelse på softwaren\cite{Sommerville}. 
Givet disse situationer, har agile metoder fungeret fordelagtigt. Grunden er, at kommunikationen mellem interessenter er konstant og uformel.

Tidligere blev det nævnt, at agile metoder ikke altid har været mainstream-metoden for softwareudvikling. En af de første populære udviklingsmetoder var The Waterfall Model.

\section{The Waterfall Model}
De første software udviklingsmodeller omhandlede en proces, der bestod af et antal faser, som repræsenterede hvordan udviklingen af software skulle foregå\cite{Sommerville}. 

\begin{figure}
    \includegraphics[width=\linewidth]{figures/waterfall_model.png}
    \caption{The Waterfall Model}
    \label{fig:waterfall}
\end{figure}

På figur \ref{fig:waterfall} ses det, at processen har en struktur som afspejler et vandfald, og herfra kommer navnet for The Waterfall Model. Princippet bag modellen er, at teamet planlægger alle aktiviteterne fremadrettet, før udviklingen af softwaren begynder. Herunder ses det at vandfaldsmodellens faser reflekterer de forskellige aktiviteter i traditionel software udvikling.\cite{Sommerville} 
\begin{itemize}
    \item Kravanalyse og definition
    \item System og software design
    \item Implementation og unit tests
    \item Integration og system tests
    \item Kontrol og vedligeholdelse 
\end{itemize}

Ved brug af vandfaldsmodellen, bør man ikke gå videre til næste fase, før den tidligere fase er færdiggjort og dokumenteret. Desuden går man heller ikke tilbage til en tidligere fase for at rette eller ændre nogle beslutninger, hvilket betyder man ofte kan blive låst fast på en dårligere løsning. Problemet ved at gå til software udvikling med denne model, er at gennem udviklingen af et helt projekt vil der altid være et tidligere problem at identificere. Hvad der menes med dette er, at hvis f.eks et team der er i gang med design, vil hurtigt identificere problemer med krav, da det ofte ikke stemmer overens. Til gengæld, vil denne ikke være dårlig for hardware udvikling. Så, nyt information vil altid komme frem, i hver aktivitet. Det betyder, at hvis en implementation viser sig at være for kompleks at lave, og uden feedback fra kunden, kan der være store forsinkelser i software designet. Dette kan lede til et dårligt design og et halvfærdigt produkt.\cite{Sommerville} 

\section{Unified Process}
Udover de Agile metoder og Waterfall modellen, findes også en mellemvej kaldet Unified Process\cite{UnifiedProcess}. Unified Process er en iterativ udviklingsmetode, hvor fremgangsmåden hælder lidt mod Waterfall. Det vil sige at der i UP er 4 centrale faser et projekt skal igennem, herunder Inception, Elaboration, Construction og Transition. I Inception fasen er der lagt meget vægt på Business Modelling og Krav til systemet. Når disse er mere eller mindre på plads kan man gå igang med Elaboration fasen, som indeholder analyse og design, implementation, tests og påbegyndelse af deployment. Ligesom i SCRUM er der iterationer eller sprints hvor man sigter efter at implementere et bestemt antal features i systemet, men i modsætning til SCRUM er der i UP også mere fokus på dokumentation og design før implementation. Når fundamentet af systemet er designet og implementeret kan man gå videre til Construction fasen hvor der er mest fokus på design, implementation og test af systemet. Til sidst er Transition hvor det sidste af systemet skal deployes ved kunden, og design gerne skulle være færdigt, såvel som de sidste implementationer af features og tests. 
Det kan både være en fordel og en ulempe at lave meget dokumentation på udviklingen af systemet. Fordelene er at der er en klar idé om, hvad systemet skal kunne, samt hvordan det skal være designet. Ulempen kan være, at hvis man ikke allerede på forhånd ved præcis hvad brugeren vil have systemet til at kunne, så kan det være svært og uoverskueligt at lave og ændre dokumentationen mange steder og ofte. Det er ikke uhørt at kunde og konsulent forstår den samme idé forskelligt. Fra tidligere projekt gik meget af udviklingstiden på dokumentation og iterationer af systemets design, men nu hvor disse designmønstre er kommet mere under huden er det ikke lige så relevant at bruge så meget tid på dokumentation og design. Udover dette ligger meget stadig uklart ift. hvordan QWest skal opereres.

\section{Metode Valg}
Nu hvor 3 metoder er kort beskrevet, kan gruppen vurdere om der skal udvikles i vandfaldsmetoden, UP eller en agil metode. Før dette er det vigtigt at pointere, at gruppen har en begrænset tidsperiode på lidt over 2 måneder til at færdiggøre projektet, som omhandler en web-applikation med samtidighed og en konsol applikation.

Eftersom teamet kun er på 4 personer og målet med projektet er at udvikle en relativ lille applikation, er det besluttet, at applikationen bliver udviklet med agile metoder. Dette er i sig selv ikke grund nok, for at vælge netop det agile over vandfaldsmetoden og UP. Derfor, er det vigtigt at opveje alle metoderne. Dette vil hjælpe med bedre at forstå, hvorfor agile metoder fordelagtigt kan vælges over vandfaldsmetoden og UP, specielt gældende dette projekt. Et af de fundamentale grundlag for at vælge agil metode i modsætning til vandfalds- eller UP metoden, er fordi applikationen er et socialt medie, som opererer på cloud-baseret-software eller bedre kendt som software as a service (saas). Saas er en metode, hvori software tillader dataens tilgængelighed til alle enheder med internet adgang og en web browser.\cite{Sommerville} Saas modeller vil aldrig virke med vandfalds baseret udvikling, da det aldrig kan blive færdiggjort, fordi det nettop kræver konstant vedligeholdelse og opdateringer.

Som nævnt tidligere, og beskrevet i Software Engineering af Ian Sommer, så benyttes agile metoder i stort set al software udvikling nu om dage, samt er de fleste apps udviklet ved brug af agile tilgange\cite{Sommerville}.  Kunden i dette projekt er ikke en ekstern kunde heller, men teamet selv, som også er involveret i udviklingen af softwaren og derfor har noget at sige ift. hvordan systemet skal se ud. I denne situation, hvor kunden er selve teamet, er det muligt at have en konstant kommunikation mellem kunden, projektlederen og udviklingsteamet. Siden kunden er teamet selv, vil der være mere feedback gennem processen, med mere præcise krav til systemet og hvordan det skal bruges. Ulempen kan være at systemets brugerflade kan blive indforstået og ikke intuitivt for nye brugere. Teamet er et lille team, og med konstant feedback vil det være nemmere at implementere konstante ændringer i kravene og softwaren til systemet. Dette skulle gerne gøres, uden at gå tilbage og lave store ændringer i dokumentation af design, før en ændring i softwaren kan indføres. Så med den agile model kan teamet hurtigt justere systemet til kundes behov. Nå det så er sagt, kan man konkludere, at den agile tilgang vil være den mest optimale tilgang, for dette projekt.

Med en beslutning om at tage den agile tilgang til projektet, skal der nu tages i betragtning hvilke udviklingsværktøjer, der skal benyttes for at skabe det bedst mulige flow for gruppen. Disse udviklingsværktøjer vil der tages hånd om i næste afsnit, som omhandler Scrum, Extreme Programming (XP) og Kanban.
