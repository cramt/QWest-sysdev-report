
\chapter{Udviklingsværktøjer}\label{ch:udviklingsværktøjer}

For at det kan lykkes med at organisere den bedste agile tilgang til projektet, skal teamet sørge for at få det bedst mulige ud af den tid og ressourcer teamet har. Dette problem bliver specielt addresseret ved Scrums agile metoder. 

\section{Scrum}


Scrum var først introduceret tilbage i 1986, af 2 japanske business eksperter, der benyttede udtrykket som en form for produkt udvikling, der skulle hjælpe med bedre organisering af stile projekter, samt skabe en form for overblik. Det var beskrevet, som en metode, der ville skabe mere fleksibilitet og hurtigere for produkt udvikling. Senere hen blev det brugt som en agil metode til at imødegå vandfalds metoder, hvor det første rigtige Scrum projekt udsprang. \cite{ScrumHistory}  Scrum Har nogle vigtige terminologier, disse er vigtig for forståelsen, for hvordan tilgangen til en Scrum projekt fungerer. 

Disse terminologier er:  
\begin{itemize}
    \item Development team, som en selvorganiserende gruppe af udviklere, deres primære ansver er at udvikle softwaren. Dette er ikke et stort team af udviklere. 
    
    \item Potentially shippable product increment, det skal altid være klar til at blive indarbejdet ind i det slut produktet. Når incrementet er færdiggjort, skal der ikke længere testes eller arbejdes videre på det.
    
    \item Produkt Backlog, er en liste af alle genstandene, teamet skal gennemføre. Et Produkt Backlog kan indeholde mange forskellige genstande, så som funktionskrav, user stories, software arkitektur og generelle opgaver for at fuldføre projektet. 

    
    \item Product owner. Jobbet som Produkt owner kan både være kunden, eller en manager, der står på at identificerer funktioner, krav, fejl og derefter prioritere alle disse ting. Produkt owner skal gennemgå Produkt Backlog, samt opdatere den hele tiden, så det møder business kravene. 
    
    \item Scrum. Hvad der menes med Scrum, er såkaldte daglige Scrum møder, hvori teamet gennemgår opgaverne, hvad der er lavet, hvad der skal laves. Dette inkluderer som reelt hele teamet, og er et forholdsvis hurtigt møde. 
    
    \item ScrumMaster. En ScrumMaster er en person der holder styr på teamet, det er lidt en mellemmand for både teamet og resten af firmaet. ScrumMasterens hoved opgave, er at teamet ikke går ud af kurs. 
    
    \item Sprint. Et Sprint er en irritation, hvori teamet udvikler alt fra funktioner til dokumentering. Disse Sprints er ofte 2 - 4 uger lange. 
    
    \item Velocity. Velocity estimerer hvilke genstande fra Product Backloget teamet kan genneføre gennem et sprint.
\end{itemize}

\section{XP}




\section{Kanban}


\section{Metode valg}


 


