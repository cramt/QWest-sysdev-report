
\chapter{Udviklingsværktøjer}\label{ch:udviklingsværktøjer}

For at det kan lykkes med at organisere den bedste agile tilgang til projektet, skal teamet sørge for at få det bedst mulige ud af den tid og ressourcer teamet har. Dette problem bliver specielt addresseret ved Scrums agile metoder. 

\section{Scrum}


Scrum var først introduceret tilbage i 1986, af 2 japanske business eksperter, der benyttede udtrykket som en form for produkt udvikling, der skulle hjælpe med bedre organisering af stile projekter, samt skabe en form for overblik. Det var beskrevet, som en metode, der ville skabe mere fleksibilitet og hurtigere for produkt udvikling. Senere hen blev det brugt som en agil metode til at imødegå vandfalds metoder, hvor det første rigtige Scrum projekt udsprang. \cite{ScrumHistory}  Scrum Har nogle vigtige terminologier, disse er vigtig for forståelsen, for hvordan tilgangen til en Scrum projekt fungerer. 

Disse terminologier er:  
\begin{itemize}
    \item Development team, som en selvorganiserende gruppe af udviklere, deres primære ansver er at udvikle softwaren. Dette er ikke et stort team af udviklere. 
    
    \item Potentially shippable product increment, det skal altid være klar til at blive indarbejdet ind i det slut produktet. Når incrementet er færdiggjort, skal der ikke længere testes eller arbejdes videre på det.
    
    \item Produkt Backlog, er en liste af alle genstandene, teamet skal gennemføre. Et Produkt Backlog kan indeholde mange forskellige genstande, så som funktionskrav, user stories, software arkitektur og generelle opgaver for at fuldføre projektet. 

    
    \item Product owner. Jobbet som Produkt owner kan både være kunden, eller en manager, der står på at identificerer funktioner, krav, fejl og derefter prioritere alle disse ting. Produkt owner skal gennemgå Produkt Backlog, samt opdatere den hele tiden, så det møder business kravene. 
    
    \item Scrum. Hvad der menes med Scrum, er såkaldte daglige Scrum møder, hvori teamet gennemgår opgaverne, hvad der er lavet, hvad der skal laves. Dette inkluderer som reelt hele teamet, og er et forholdsvis hurtigt møde. 
    
    \item ScrumMaster. En ScrumMaster er en person der holder styr på teamet, det er lidt en mellemmand for både teamet og resten af firmaet. ScrumMasterens hoved opgave, er at teamet ikke går ud af kurs. 
    
    \item Sprint. Et Sprint er en irritation, hvori teamet udvikler alt fra funktioner til dokumentering. Disse Sprints er ofte 2 - 4 uger lange. 
    
    \item Velocity. Velocity estimerer hvilke genstande fra Product Backloget teamet kan genneføre gennem et sprint.
\end{itemize}

Hvad der gør specifikt Scrum sen agile metode, er at den følger principperne, dog med fokus på at anbringe et framework, for hvordan et projekt organiseres. På figur 2.1, ses principperne for det agile manifest. Sammenlignet med terminologierne for oven er det tydeligt, at det følger det agile manifest. \cite{Sommerville} 
 

\section{XP}

Endnu en form for agile medtoder er, Extreme Programming (XP), som er en metode at lave software på, ved at tage alle de ting som virker og tage det så ekstremt som mulig, dette gør XP unik ift andre iterative udviklingsmetoder. Der findes også ligehedder mellem XP og andre, så som, at et team planlægger et stykke arbejde, som udarbejdes i løbet af nogle få uger, præcis som i Scrum, hvor man planlægger et sprint på få 2-4 uger. Et eksempel på hvordan XP tager ting til ekstremerne er, pair programming. Pair programming er en udviklings teknik, hvori 2 programmør sidder sammen om en computer. Den ene, driveren, er personen som skriver koden, den anden oberster hver enste linje af koden som driveren udvikler. Forskning har vist, at code reviews er en af de bedste metoder til ar finde fejl i koden, så derfor har XP pair programming.\cite{WhatisXP}

XP, blev først brugt i et projekt 6. Marts 1996. Siden hen, har det vist sig, at være meget succesfuld, i mange forskellige virksomheder, rundt omkring i verden. Snakkes der om XP, er der nogle bestemte værdier og principper man skal tage i betragtning, dette vil være XP’s 5 værdier og 12 praktikker. Med de 5 værdier og 12 praktikker, kan et projekt forbedres markant bedre, da XP sætter fokus i, konstant kommunikation med kunden og egne udviklere. Det er succesfuld, da det bygger på kundens tilfredstillelse, og det gør dette ved at give kunden hvad kunden har brug for, derved vil XP også skabe en hvis fleksibel rutine, hvis kunden kommer med nye krav for enden af en cyklus. 

De 5 værdier: 

\begin{itemize}
    \item Respekt, både overfor hinanden og stakeholders. 

    \item Kommunikation, som er kritisk for succes. 

    \item Simpelt som muligt, menes med at jo simplere det er, desto bedte 

    \item Feedback, som kan være fra kunder, brugere og andre. 

    \item Mod. Mod til at lave ting. Mod til at eksperimentere. Mod til at sige nej til kunderne. Det behøver ikke være formelt. 
\end{itemize}

De 12 praktikker: 



\begin{itemize}
    \item Planning game

    \item Smal realeases
  
    \item Metaphor

    \item Simple design 

    \item Define test first

    \item Refactoring
 
    \item Pair programming

    \item Collective ownership
   
    \item Continuos integration 

    \item 37 hour workweek 

    \item On-site customer 
    
    \item along Coding standarts 


\end{itemize}

XP minder generelt meget om Scrum, de kræver begge teams til at arbejde iterativt og med tæt samarbejde. Hvor de afviger er pratikker så som Continues Intergration, Refactoring og Pair Programming, som er essentielle for at få den iterative proces til at virke. Hvis en af praktikkerne i XP lider, så kommer resten til at lide. 



\section{Kanban}


\section{Metode valg}


 


