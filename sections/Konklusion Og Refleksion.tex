\chapter{Konklusion Og Refleksion}\label{ch:KonklusionRefleksion}

\section{Konklusion}
Målet med projektet var at udvikle et system som kunne bruges til at dokumentere rejseoplevelser, og skulle håndtere samtidighed mellem brugere. Udvikling af sådan et system kræver en udviklingsmetode gruppen kan følge, for at mindske risici og eller afvigelse fra kursen. Hertil analyserede vi agile metoder, vandfalds metoden og Unified Process, lagde dem op mod hinanden og fandt så den umiddelbart bedste tilgang til projektet. Vi fik konkluderet, at en agil metode til et lille projekt var den bedste tilgang. 

Sammenlignet med et tidligere projekt som foregik med UP og plandreven udvikling, kan gruppen konkludere, at den agile udviklingsmetode har været fordelagtig. Kombinationen af Scrum og XP har vist, at udviklingstilgangen har været nyttig til at give et overblik over hvilke features, der skal implementeres ud fra kunde-værdi, og måden hvorpå målene for projektet opdateres løbende har også sørget for at holde værdi for kunden i fokus. Daily Scrum og 2 ugers Sprints har hjulpet gruppen i en høj grad med ikke at komme ud af kurs, derved undgå risici. 
I løbet af første sprint blev der oprettet en basis arkitektur, som senere blev videreudviklet til at understøtte hele systemet med de features der senere blev implementeret. Arkitekturen var designet med henblik på videreudvikling, og det lykkedes også at få det sådan.

Elementerne fra XP, såsom TDD, Pair Programming, skabte et bedre fundament for systemet under udviklingen. TDD betød at kode virkede før det blev implementeret, hvilket sparede forvirring og tid, og Pair Programming gav bedre forståelse af hele programmets sammensætning for alle i gruppen.

Det lykkedes os i dette projekt at finde en løsning på vores Problemformulering, da vi netop fik oprettet et system som kunne dokumentere rejseoplevelser, dele minder med venner og håndtere samtidighedsproblemer mellem flere brugere af systemet.

 
 


\section{Reflektion}
