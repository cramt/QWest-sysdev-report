\chapter{Udviklingen}\label{ch:Udviklingen}

Nedstående sektioner vil sætte brugen af agile metoder i fokus. 

\section{Brug af Scrum}

For at danne et bedre overblik over hvordan gruppen har benyttet Scrum, kan vi se på hvordan de 9 forskellige Scrum roller, ceremonier(+2) og artifakter er benyttet, i den nævnte rækkefølge.

\subsection{Roller}

Produkt ejeren i dette projekt har været teamet selv, nettop fordi det er et lille team, som er uerfaren med hvordan et Scrum projekt skal kunne fungerer. Siden hen teamet har valgt at håndtere rollen som produkt ejeren, betyder dette at alle har haft en indflydelse på definitionen af user stories. Det samme gælder priortiseringen af produkt backloggen, samt andre tekniske funktioner af produktet. 

Vedrørende Scrum master er det samme princip ikke brugt, som produkt ejeren. Hvad der menes med dette, er at Scrum master denne gang har været en specifik person, dog med en rokering for hvert sprint. 

Nu da de 2 første roller er gennemgået, er det denne gang udviklings teamet, som skal tages i overvejelse. Med en gruppe på blot 4 personer, har beslutningen for denne rolle været, at hele teamet er en del af udviklings teamet. Det menes, at denne beslutning vil skabe den bedste værdi for hver enkelt individ, så personerne alle har et overblik over hvordan arkitekturen er udviklet. 



\subsection{Ceremonier}

Teamet startede projektet ud med at danne 8 forskellige user stories, som blev benyttet ti at danne et produkt backlog, hvori prioriterering, ID og korte beskrivelser af ønskede funktioner til produktet er beskrevet. Ud fra denne produkt backlog har teamet nu kunne oprette et sprint backlog, med et henblik hvad der er prioriteret højst på produkt backloggen. Herfra har teamet oprettet en liste over alle opgaver der skal udføres, med et estimat af tidsfordeling på opgaver ved hjælp af planning poker. Sprint backloggen er blevet opdateret hver eneste dag. Nederest er det muligt at 3 forskellige eksempler, der viser teamets egen produkt backlog, sprint backlog og det burn down chart teamet har oprettet for at illustrerer fremskidt, nemlig hvor langt man er i forhold til planen. 

\subsection{Artifakter}

I sprint 0 sørgede teamet for at danne et godt overblik, over hvilke opgaver der havde den højste prioritet. Det valgte egenskaber blev så tilføjet til sprint backloggen, selvfølgelig henblik på hvor stor en opgave det er, så teamet ikke blev overvældet af mængden af opgaver. Efter denne proces, var det tid til at estimerer opgaverne. Dette blev gjort i brug af planning poker. Når så alle opgaverne var estimeret og fordelt til forskellige team medlemmer, blev projektet sat i gang. Herfra har teamet ydet for at opnå såvidt som det nu kunne lade sig gøre, at følge ceremonierne. De daglige Scrum møder, har ikke altid været daglige, på grund af undervisning eller forglemmelse af at informerer teamet, om mødetidspunkt, men teamet har til gengæld været gode til at opdaterer på processen overfor hindanden. For hvert nyt sprint, har denne proces gentaget sig, med opdateringer på produkt backlog og en reflektion / evaluering af sprintet. 



\section{Brug af XP med Scrum}



\section{Something something}