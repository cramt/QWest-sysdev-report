\chapter{Systemvision}\label{ch:systemvision}

\section{Formål og afgrænsning}\label{sec:purpose}
Projektet afgrænses til at udvikle et brugervenligt system hvor brugere kan markere hvor de har været i verden, og dele oplevelser med venner.

\section{Problemområde}\label{sec:problemarea}
Mange gange når man gerne vil vise billeder fra oplevelser man har haft, kan det være svært at huske præcis, hvor man var henne på det tidspunkt, og hvordan det ligger ift. resten af verden. Når man scroller ned igennem kamerarullen er det bare en masse forskellige steder uden kontekst. Dette kan gøre det svært at sætte sine oplevelser i perspektiv, specielt når man vil fortælle andre omkring det, eller danne sig overblik over sine oplevelser. Der vil også være metadata omkring sine oplevelser, såsom dato (måske bare årstal og årstid?), hvad man lavede, måske en sjov oplevelse der skete lige efter/før billedet blev taget, osv. Nogle rejser til de samme steder igen og igen, uvidende omkring hvad der ellers er af seværdigheder og områder, der også kunne være spændende. Man kan så forbinde sine oplevelser til et kort, og de rejselystne kan nemt se hvor de mangler at rejse til.

\section{Problemformulering}\label{sec:problemstatement}
Hvordan bygger vi bedst et system, som kan dokumentere ens rejseoplevelser på en overskuelig måde og sætter de minder man deler i bedre kontekst, samt hvordan håndterer vi bedst muligt at flere brugere kan redigere i samme tekst under hver sektion af landkortet?

\section{Interessanter og brugere}\label{sec:users}
Projektet er målrettet mod rejse-glade mennesker, som gerne vil dele deres rejseoplevelser og se de steder de har været. Det er også målrettet grupper som rejser sammen og gerne vil være sammen om at dokumentere rejsen.

\section{Teknologi}\label{sec:technology}
Alle teknologi og programmeringsovervejelser forefindes i programmering- og teknologirapporten.