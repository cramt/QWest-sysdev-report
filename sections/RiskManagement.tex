\chapter{Risk Management}\label{ch:Risk Management}

Når det gælder software projekter, eller i det hele taget projekter generelt, er der flere risici, som bør identificeret, overvejet og analyseret. Risici er mulige negative afvigelser fra planen.\cite{SlideRiskAnalysis} I dette afsnit, vil gruppen danne et overblik over forskellige metoder, som kan bruges til at identificere risici, samt hvordan risici er blevet analyseret ift. projektet.

\section{Risici}

Der er mange mulige metoder til at analysere hvilke risici et team står overfor. Fælles for dem er, at starte ud med at identificere mulige risici.
Identificering af ricisi kan blive skåret ned i meget mindre steps. \cite{SlideRiskAnalysis} Disse steps fokuser på at stille en række sprøgsmål ind til hvilket slags projekt det er, hvor mange der kommer til at benytte det og hvad der vil ske hvis det ikke virker? Hertil, reflekterer over tidligere projekter, om der var noget der teamet burde tage med? Og til sidst, beskrive det i ‘cause, event and effect’. Identificerings processen er i sig selv, en brainstorming process.

Det næste punkt, efterfulgt af at identificere, er vurderer risici. Vurdering af risici gennemgår 2 punkter: Det ene estimering, og det andet evaluering. Estimering dækker sandsynligheden for risici, hvilken effekt det vil have for både trusler og muligheder. Evaluering er når man samler alle risici og danner en risiko værdi for hele projektet.

%Husk at det centrale i dette kapitel er hvilke risici vi har identificeret - dernæst hvilke metoder, og hvordan risici er håndteret.

\section{Boehm}

Boehm skemaet danner et bedre overblik over det metode valg, som teamet har valgt at benytte. Da det ikke altid er den agile metode, som er det bedste valg til et projekt er det vigtigt at identificere hvilke ricisi, der kan opstå ved nettop at benytte den agile tilgang. En fejl teamet kan begå, er automatisk gå ud fra at en agil tilgang er det bedste valg. Denne fejl vil vi i gruppen helst ikke begå, så derfor har vi benyttet Boehm skemaet, til at få den bedste forståelse til hvor gruppen ligger, i forhold til projektet. 
Nu hvor teamet er på blot 4 personer og størrelsen på personel er meget lille, vil disse to punkter ligge tættere på midten. Vi er i et forholdvis behageligt projekt, da hvis noget går galt, vil dette hverken koste et eller flere liv. Kigger vi i det hele taget på Boehm skemaet, bliver det hurtigt tydeligt, at teamet i hverfald hører ind under de røde linjer og derfor passer bedst ind i den agile tilgang. 

\section{Identificerede risici}
