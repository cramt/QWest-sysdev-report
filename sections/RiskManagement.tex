\chapter{Risk Management}\label{ch:Risk Management}

Når det gælder software projekter, eller i det hele taget projekter generelt er der flere risici, som bør blive overvejet og analyseret. Risici er mulige negative afvigelser fra planen.\cite{Slide2} I dette afsnit, vil gruppen danne et overblik over det forskellige metoder risici kan blive identificeret, samt hvordan risici er blevet analyseret ift. projektet. 

\section{Risici}
  
Der er mange mulige metoder at analyserer hvilke risici et teams står overfor, hvad der er grundlæggende for de fleste af dem er, at starte ud med at identificere mulige risici. 
Identificering af ricisi kan blive skåret ned i meget mindre steps. \cite{Slide2} Disse steps fokuser på at stille en række sprøgsmål ind til hvilket slags projekt det er, hvor mange der kommer til at benytte det og hvad der vil ske hvis det ikke virker? Hertil, reflekterer over tidligere projekter, om der var noget der teamet burde tage med? Og til sidst, beskrive det i ‘cause, event and effect’. Identificerings processen er i sig selv, en brainstorming process. 

Det næste punkt, efterfuldt af at identificere, er vurderer risici. Vurdering af risici gennemgår 2 punkter: Det ene estimering, og det andet evaluering. Estimering dækker sandsynligheden for risici, hvilken effekt det vil have for både trusler og muligheder. Evaluering er når man samler alle risici og danner en risiko værdi for hele projektet. 



\section{Boehm}


\section{Hvad er gjort for at få det løst?}